\section*{Abstract}
Continuous development of quadcopters in recent years has led to their wide application in many industries and research fields%. Due to their simple mechanical and electronic construction and the smaller, cheaper, more precise sensor and computer units they are relatively inexpensive. In addition, their good maneuverability and flight capabilities allow them to be used in many applications
, including rescue missions, agricultural monitoring, making camera footages or creation of spectacular entertainment shows.
As the application fields are continuously widening, more and more capabilities are expected from the drones. Agile maneuvering in a cluttered environment, working in team, cooperating with humans or performing acrobatic maneuvers, just to name a few. These tasks require to perform complex, fast maneuvers that pushes the drones to their physical limits. Classical flight controllers based on a linearized dynamical model are no longer applicable for these tasks and more advanced control methods that exploit the entire operating domain are needed. These control algorithms can be developed by mathematically well-grounded nonlinear design techniques (e.g. differential geometry) or machine learning approaches.

In this work, trajectory planning and motion control design to execute a flip maneuver is presented for a nano quadcopter. Such a maneuver is a challenging task even for experienced drone pilots. The complexity and speed of the maneuver is characterized by the fact that it takes less than a second to complete, during which the vehicle is able to make a full turn around one of the horizontal axes.

After presenting the theoretical background and relevant literature, two approaches are proposed to perform the flip maneuver. The first is a simple, open-loop strategy which is designed by optimizing the sequence of specific motion primitives describing the flip maneuver using a digital twin model of the quadcopter. The second approach is based on a nonlinear geometric controller, designed by using a dynamical model of the drone, that is able to track even highly challenging reference trajectories. To perform the flip maneuver, first a feasible reference trajectory is designed that describes the intended state evolution. Then, the designed trajectory is precisely tracked by the geometric controller. Finally, the two approaches are compared in both simulation and real-world implementation based measurement results.
\pagebreak