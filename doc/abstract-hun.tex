\begin{otherlanguage}{magyar}
\section*{Összefoglaló}
A kvadkopterek folyamatos fejlődése az elmúlt évek során ahhoz vezetett, hogy ma már számos területen alkalmazzák őket az iparban és kutatások során egyaránt, mint például felderítés mentőakciók során, kamerafelvételek készítése hírközlés céljából, illetve mezőgazdasági monitorozás. Alkalmazási területeik folyamatos bővülésével a kvadkoptereknek egyre több képességgel kell rendelkeznie, mint agilis manőverezés akadályok közötti szűk helyeken, kooperáció több drónnal, emberekkel való együttműködés, illetve akrobatikus manőverek bemutatása. Ilyen feladatok végrehajtásához gyors, komplex manőverezésre van szükség a drón fizikai határait teljes mértékben kihasználva. Ehhez a linearizált dinamikai modellre tervezett klasszikus szabályozók nem alkalmazhatóak, komplexebb irányítási stratégiára van szükség, ami az egész működési tartományon képes szabályozni a kvadkopter mozgását. Ilyen algoritmusok tervezéséhez felhasználhatóak nemlineáris irányítási módszerek, például differenciálgeometria vagy gépi tanulás alapján.

%Leggyakrabban a kvadkopterek irányításának célja, hogy eljussanak egy kiinduló helyzetből egy célpozícióba, miközben elkerülik az esetleges akadályokat. Ez a navigációs feladat két részre bontható: a pálya megtervezésére, illetve a trajektóriakövetésre. Sokszor a pálya követésére megfelelőnek bizonyul a drón linearizált modelljére tervezett lineáris szabályozó alkalmazása, mivel nem szükséges nagy szögkitérésekkel operálni. Adódnak azonban olyan helyzetek, ahol hirtelen, gyors manőverek végrehajtása szükséges, amit egy komplexebb irányítási struktúrával lehet csak végrehajtani. Az ilyen szabályozók jellemzően nemlineárisak, előállhatnak például robusztus irányítási módszerek alapján, vagy neurális háló struktúrában.

Dolgozatom során egy bonyolult, tapasztalt emberi pilóták számára is kihívást jelentő manőver, a bukfenc pályatervezését és pályakövető szabályozását mutatom be egy kisméretű kvadkopteren szemléltetve. A manőver komplexitását és gyorsaságát jellemzi, hogy a végrehajtása kevesebb, mint egy másodpercet vesz igénybe, ami alatt a jármű egy teljes fordulatot képes megtenni.

Az elméleti háttér és releváns irodalom bemutatása után a manőver végrehajtására két különböző megközelítést demonstrálok. Az első egy nyílt hurkú irányítási stratégia a bukfencet leíró paraméterezett primitívekből álló szekvenciának a drón digitális ikermodelljén való optimalizációja alapján. A második egy nemlineáris geometriai elvű szabályozás a kvadkopter dinamikai modelljére tervezve, ami igen bonyolult referenciapálya követésére is képes. A bukfenc manőver végrehajtásához először egy alkalmas referenciapálya tervezése szükséges specifikálva a kívánt mozgás állapotait, amit a kvadkopter geometriai pályakövető szabályozással precízen tud követni. Végül a két irányítási stratégia összehasonlítását mutatom be szimulációs és a valós rendszeren végzett mérési eredmények alapján.
\pagebreak
\end{otherlanguage}